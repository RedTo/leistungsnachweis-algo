\href{https://travis-ci.org/ob-algdatii-ss18/leistungsnachweis-algo}{\tt !\mbox{[}Build Status\mbox{]}(https\-://travis-\/ci.\-org/ob-\/algdatii-\/ss18/leistungsnachweis-\/algo.\-png?branch=master)} \href{https://coveralls.io/github/ob-algdatii-ss18/leistungsnachweis-algo?branch=master}{\tt !\mbox{[}Coverage Status\mbox{]}(https\-://coveralls.\-io/repos/github/ob-\/algdatii-\/ss18/leistungsnachweis-\/algo/badge.\-png?branch=master)}

\section*{Sudoku}

\href{https://ob-algdatii-ss18.github.io/leistungsnachweis-algo/}{\tt Projekthomepage} ⋅ \href{https://ob-algdatii-ss18.github.io/leistungsnachweis-algo/doxygen_doc/html/index.html}{\tt Dokumentation} ⋅ \href{https://ob-algdatii-ss18.github.io/leistungsnachweis-algo/}{\tt Sudoku ausprobieren} ⋅ \href{https://coveralls.io/github/ob-algdatii-ss18/leistungsnachweis-algo/}{\tt Coveralls.\-io} ⋅ \href{https://travis-ci.org/ob-algdatii-ss18/leistungsnachweis-algo}{\tt Travis-\/ci.\-org} \subsection*{Thema}

Umgesetzt werden soll ein System, dass Sudokus generiert. Dabei wird im ersten Schritt das normale Sudoku erstellt. Später kann das Sudoku skalieren durch neue Modi wie bestimmte Teilsummen, Vergrößerung des Zahlenbereich oder alternative Lösungsfindung. \subsection*{Dokumentation}

Die Dokumentation und weiterführende Informationen zu unserem Projekt sind \href{https://ob-algdatii-ss18.github.io/leistungsnachweis-algo/}{\tt auf dieser Seite} zu finden. 